\documentclass[12pt, titlepage]{article}
\usepackage{fullpage}
\usepackage[round]{natbib}
\usepackage{multirow}
\usepackage{booktabs}
\usepackage{tabularx}
\usepackage{graphicx}
\usepackage{float}
\usepackage{hyperref}
\hypersetup{
    colorlinks,
    citecolor=black,
    filecolor=black,
    linkcolor=red,
    urlcolor=blue
}
\usepackage[round]{natbib}
\newcounter{acnum}
\newcommand{\actheacnum}{AC\theacnum}
\newcommand{\acref}[1]{AC\ref{#1}}
\newcounter{ucnum}
\newcommand{\uctheucnum}{UC\theucnum}
\newcommand{\uref}[1]{UC\ref{#1}}
\newcounter{mnum}
\newcommand{\mthemnum}{M\themnum}
\newcommand{\mref}[1]{M\ref{#1}}
\title{SE 3XA3: Software Requirements Specification\\Title of Project}
\author{Team \#10, Ben 10
		\\ Rehan Theiveehathasan \texttt{theivers}
		\\ Karnvir Bining                  \texttt{biningk}
		\\ Puru Jetly                        \texttt{jetlyp}
}
\date{\today}

\begin{document}
\maketitle
\pagenumbering{roman}
\tableofcontents
\listoftables
\listoffigures
\begin{table}[bp]
\caption{\bf Revision History}
\begin{tabularx}{\textwidth}{p{3cm}p{2cm}X}
\toprule {\bf Date} & {\bf Version} & {\bf Notes}\\
\midrule
November 13, 2016 & 1.0 & Initial changes\\

\bottomrule
\end{tabularx}
\end{table}
\newpage
\pagenumbering{arabic}
\section{Introduction}
This is the Design Documentation(DD-Rev0) for Ben 10’s project Pong Invaders. This document outlines the design specifications as outlined by Dr. Spencer Smith.
Complimentary documentation includes a Module Interface Specifications(MIS) document written with Javadoc.  Potential readers of this document include:
\begin{itemize}
\item Designers/Developers: This document is a way to communicate how the program being designed should function in terms of coupling and meeting requirements. It allows developers to stay on the same page and consistent with design assumptions. The breakdown of module interaction and traceability to requirements allows future developers to easily distinguish areas that can be optimized or changed. 
\item New Members: This documentation is an effective way to portray the project focus and aim to new members who need to be brought up to speed. 
\end{itemize}

\section{Anticipated and Unlikely Changes} \label{SecChange}

\subsection{Anticipated Changes} \label{SecAchange}

\begin{description}
\item[\refstepcounter{acnum} \actheacnum \label{acHardware}:] 
\item[\refstepcounter{acnum} \actheacnum \label{acInput}:] 
\item ...
\end{description}
\subsection{Unlikely Changes} \label{SecUchange}

\begin{description}
\item[\refstepcounter{ucnum} \uctheucnum \label{ucIO}:] 
\item[\refstepcounter{ucnum} \uctheucnum \label{ucInput}:] 
\item ...
\end{description}
\section{Module Hierarchy} \label{SecMH}

\begin{description}
\item [\refstepcounter{mnum} \mthemnum \label{mHH}:]
\item ...
\end{description}
\begin{table}[h!]
\centering
\begin{tabular}{p{0.3\textwidth} p{0.6\textwidth}}
\toprule
\textbf{Level 1} & \textbf{Level 2}\\
\midrule
{Hardware-Hiding Module} & ~ \\
\midrule
\multirow{7}{0.3\textwidth}{Behaviour-Hiding Module} & ?\\
& ?\\
& ?\\
& ?\\
& ?\\
& ?\\
& ?\\ 
& ?\\
\midrule
\multirow{3}{0.3\textwidth}{Software Decision Module} & {?}\\
& ?\\
& ?\\
\bottomrule
\end{tabular}
\caption{Module Hierarchy}
\label{TblMH}
\end{table}
\section{Connection Between Requirements and Design} \label{SecConnection}

\section{Module Decomposition} \label{SecMD}

\subsection{Hardware Hiding Modules (\mref{mHH})}
\begin{description}
\item[Secrets:]
\item[Services:]
\item[Implemented By:] OS
\end{description}

\subsection{Behaviour-Hiding Module}
\begin{description}
\item[Secrets:]
\item[Services:]
\item[Implemented By:] --
\end{description}

\subsubsection{Input Format Module (\mref{mInput})}
\begin{description}
\item[Secrets:]
\item[Services:]
\item[Implemented By:] Pong Invaders
\end{description}

\subsubsection{Etc.}
\subsection{Software Decision Module}
\begin{description}
\item[Secrets:] 
\item[Services:] 
\item[Implemented By:] --
\end{description}
\subsubsection{Etc.}

\section{Traceability Matrix} \label{SecTM}
\begin{table}[H]
\centering
\begin{tabular}{p{0.2\textwidth} p{0.6\textwidth}}
\toprule
\textbf{Req.} & \textbf{Modules}\\
\midrule
R1 & \mref{mHH}, \mref{mInput}, \mref{mParams}, \mref{mControl}\\
R2 & \mref{mInput}, \mref{mParams}\\
R3 & \mref{mVerify}\\
R4 & \mref{mOutput}, \mref{mControl}\\
R5 & \mref{mOutput}, \mref{mODEs}, \mref{mControl}, \mref{mSeqDS}, \mref{mSolver}, \mref{mPlot}\\
R6 & \mref{mOutput}, \mref{mODEs}, \mref{mControl}, \mref{mSeqDS}, \mref{mSolver}, \mref{mPlot}\\
R7 & \mref{mOutput}, \mref{mEnergy}, \mref{mControl}, \mref{mSeqDS}, \mref{mPlot}\\
R8 & \mref{mOutput}, \mref{mEnergy}, \mref{mControl}, \mref{mSeqDS}, \mref{mPlot}\\
R9 & \mref{mVerifyOut}\\
R10 & \mref{mOutput}, \mref{mODEs}, \mref{mControl}\\
R11 & \mref{mOutput}, \mref{mODEs}, \mref{mEnergy}, \mref{mControl}\\
\bottomrule
\end{tabular}
\caption{Trace Between Requirements and Modules}
\label{TblRT}
\end{table}
\begin{table}[H]
\centering
\begin{tabular}{p{0.2\textwidth} p{0.6\textwidth}}
\toprule
\textbf{AC} & \textbf{Modules}\\
\midrule
\acref{acHardware} & \mref{mHH}\\
\acref{acInput} & \mref{mInput}\\
\acref{acParams} & \mref{mParams}\\
\acref{acVerify} & \mref{mVerify}\\
\acref{acOutput} & \mref{mOutput}\\
\acref{acVerifyOut} & \mref{mVerifyOut}\\
\acref{acODEs} & \mref{mODEs}\\
\acref{acEnergy} & \mref{mEnergy}\\
\acref{acControl} & \mref{mControl}\\
\acref{acSeqDS} & \mref{mSeqDS}\\
\acref{acSolver} & \mref{mSolver}\\
\acref{acPlot} & \mref{mPlot}\\
\bottomrule
\end{tabular}
\caption{Trace Between Anticipated Changes and Modules}
\label{TblACT}
\end{table}

\section{Use Hierarchy Between Modules} \label{SecUse}
\begin{figure}[H]
\centering
%\includegraphics[width=0.7\textwidth]{UsesHierarchy.png}
\caption{Use hierarchy among modules}
\label{FigUH}
\end{figure}
%\section*{References}
\bibliographystyle {plainnat}
\bibliography {MG}
\end{document}