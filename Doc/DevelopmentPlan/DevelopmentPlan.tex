\documentclass{article}

\usepackage{booktabs}
\usepackage{tabularx}
\usepackage{hyperref}
\title{SE 3XA3: Development Plan\\Title of Project}
\author{Team \#10, Ben Ten
		\\ Rehan Theiveehathasan	\texttt{theivers}
		\\ Karnvir Bining	\texttt{biningk}
		\\ Puru Jetly	\texttt{jetlyp}
}
\date{}
\begin{document}
\begin{table}[hp]
\caption{Revision History} \label{TblRevisionHistory}
\begin{tabularx}{\textwidth}{llX}
\toprule
\textbf{Date} & \textbf{Developer(s)} & \textbf{Change}\\
\midrule
September 29, 2016 & Puru Jetly & Intial changes\\
September 29, 2016 & Karnvir Bining & Intial changes\\
September 29, 2016 & Rehan Theiveehathasan & Intial changes\\
\bottomrule
\end{tabularx}
\end{table}
\newpage
\maketitle
Put your introductory blurb here.
\section{Team Meeting Plan}

When: 8 - 10:30 PM \newline
\newline
Where: On skype or at Thode\newline
\newline
Frequency: On Wednesday and Saturday\newline


\noindent Experts: \newline
\indent Documentation: Karnvir Bining\newline
\indent Git: Rehan Theiveehathasan\newline
\indent LaTeX: Puru Jetly\newline
\indent Technology: \newline
\indent  \indent Java: All members \newline

\noindent Rules for Agenda:\newline

\noindent There will be a chair for each meeting. The person who is chair for the meeting will alternate on a week by week basis. The order for chair will be Karnvir, Puru, Rehan. Since there will be two meetings in a week, the first meeting will be open discussion, and the second one will be structured to address and finalize the issues from open discussion.In open disscusion, the group will discuss what deliverables are due by next meeting  and how many resources should be dedicated to each task. In the second meeting, the deliverables will be handed in, and if a deliverable was not complete more resources will be used to complete the deliverable before submission day.

\section{Team Communication Plan}
Majority of team communication will be done through facebook, and over text messaging. For certain git related problems we will communicate through %% you have a typo here: the ’ character is NOT the ' character, and LaTeX doesn't render the ’ correctly
git’s issue tracking and through well commented commits. This will allow for the team to stay on the same page even when communication between members is not direct. If a team member is unreachable for some reason an alternate contact will also be provided. In most cases this contact is a roommate, whom we can contact to resolve the issue. \newline \newline
\begin{table}[h!]
  \caption{Team Communication Table}
  \label{tab:table1}
  \begin{tabular}{|l|c | c |r|}
    \hline
    \textbf{Name} & \textbf{Phone} & \textbf{Alt. Contact Name} & \textbf{Alt. Contact Phone}\\
    \hline
    Puru J. & 647 - 772 - 5409 & Sebastian T. & \multicolumn{1}{c|}{ 647 - 378 - 7162 }\\
    \hline
    Rehan T. & 647 - 993 - 8561 & David H. & \multicolumn{1}{c|}{ 519 - 546 - 6165}\\
    \hline
    Karnvir B. & 647 - 717 - 6969 & Ramanan R. & \multicolumn{1}{c|}{ 647 - 631 - 8566}\\
    \hline
  \end{tabular}
\end{table}
\section{Team Member Roles}
Team Leader: Puru Jetly\newline
Scribes: Karnvir Bining , Rehan Theiveehathasan

\noindent * Scribes will alternate meeting by meeting\newline
\section{Git Workflow Plan}
\indent The centralized Git workflow will be used for developing Pong Invaders. This will allow all three developers for the game to be able to consistently update, and build the project while maintaining a modular design. Labels will be implemented through git tags, and are to be used to see when milestone commits are made. Milestones will be used to mark when new functionality, or significant updates are implemented. In tandem with labels, and milestones, issue tracking will be used to open issues for modules to be completed, to be debugged, and to keep track of what milestones need to be implemented still according to the Gantt chart.\newline
\section{Proof of Concept Demonstration Plan}

Implementing the game into MVC architecture is not difficult, as all group members are familiar with this architecture. However after the implementation of Space Invaders, overlaying Pong may pose a difficulty.Such as having modules interfering with each other, and not having ample time to implement the new Pong modules.The way this difficulty can be overcome is by modularizing the Space Invaders program, and by keeping ahead of schedule for the implementation of Space Invaders. Testing games are difficult as there are many test cases that need to be used to ensure the system is completely free of critical bugs. JUnit testing will be used to check the validity of the program. 

\section{Technology}
\indent The programming language of choice for this project will be Java with Eclipse being the IDE of choice. Java was chosen specifically as the source code was also compiled in Java. Java code can also be run on any machine that currently has Java installed. Git will be used for versioning to keep track of the Pong Invader source code while it is being developed. Makefiles will also be used, they will be used to compile the source code, thus a bash terminal is necessary.
\section{Coding Style}
%% same apostrophe problem here.
\indent Pong Invaders’ coding style will follow standard Java naming conventions, as seen here: http://www.oracle.com/technetwork/java/codeconventions-135099.html. Pong Invaders will also use tabs instead of spaces, and have comments for every block of code or method. This coding style is easy to read, and very easy to follow.
\section{Project Schedule}
\url {https://gitlab.cas.mcmaster.ca/biningk/Space_Invaders_10/blob/master/Doc/DevelopmentPlan/Pong_Invaders_Project_Schedule.pdf}
%\section{Project Review}
\newpage

\end{document}